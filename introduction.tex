%-----------------------------------------------------------------------------------------------
\chapter*{Bevezetés}\addcontentsline{toc}{chapter}{Introduction}
%-----------------------------------------------------------------------------------------------

Ez a dokumentum a 2016 őszi félévében, Önálló laboratórium 2 tárgy keretei között végzett munkám összefoglalója.

Az itt közölt eredmények építenek az előző féléves, azonos témában végzett kutatásomra.
Akkor a feladat a strukturált fényt használó 3D rekonstrukciós eljárások vizsgálata volt a cél.
Az elvek gyakorlati kipróbálására a Microsoft Kinect adott kiváló platformot.
Az előző féléves munka legjavát a technológiával és módszerekkel való ismerkedés adta.
A Kinect által szolgáltatott infravörös kép elemzésével próbáltam reprodukálni az eszköz belső működését.

Az előző félév munkája proof-of-cocept jellegű volt.
A mostani ezen túlmutat.
A cél most kettős: egy hosszútávon használható, rugalmas, moduláris keretrendszer fejlesztése a diszparitás meghatározásához, valamint rekonstrukció minőségének javítása a kép lokális struktúrájának figyelme vételével.