%-----------------------------------------------------------------------------------------------
\chapter*{Bevezetés}\addcontentsline{toc}{chapter}{Introduction}
%-----------------------------------------------------------------------------------------------

Ez a dokumentum a 2016 őszi félévében, Önálló laboratórium 2 tárgy keretei között végzett munkám összefoglalója.

Az itt közölt eredmények építenek az előző féléves, azonos témában végzett kutatásomra.
Akkor a feladat a strukturált fényt használó 3D rekonstrukciós eljárások vizsgálata volt.
Az elvek gyakorlati kipróbálására a Microsoft Kinect adott kiváló platformot.
Az előző féléves munka legjavát a technológiával és módszerekkel való ismerkedés adta.
A Kinect által szolgáltatott infravörös kép elemzésével próbáltam reprodukálni az eszköz belső működését.

Az előző félév munkája proof-of-concept jellegű volt.
A mostani ezen túlmutat.
A cél most kettős: egy hosszútávon használható, rugalmas, moduláris keretrendszer fejlesztése a diszparitás meghatározásához, valamint rekonstrukció minőségének javítása a kép lokális struktúrájának figyelembe vételével.

Az első fejezetben röviden összefoglalom a használt algoritmusokat és paraméterezésüket.
Ez részben az előző féléves munka összefoglalása is.

A második, egyben leghosszabb fejezet tartalmazza a fejlesztett keretrendszer leírását.
Ismertetésre kerül a program felhasználó felülete, valami a programozási struktúra és a fejlesztői interfész is.

A harmadik fejezet a lokális struktúra figyelembevételével foglalkozik.
Az itt tárgyalt algoritmusok kísérleti jellegűek, a későbbiekben behatóbb vizsgálatot és optimalizációt igényelnek.

A negyedik részegység az eredmények rövid összegzését és néhány példát tartalmaz.
