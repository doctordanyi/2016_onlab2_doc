%-----------------------------------------------------------------------------------------------
\chapter{Lokális struktúra vizsgálata}\label{sect:LocalStruct}
%-----------------------------------------------------------------------------------------------

Az illeszkedési minőség bizonyos esetekben javítható, ha figyelembe vesszük a kép lokális struktúráját.
Ez a megoldás rontja az általánosságot, mert extra megkötéseket követel meg a diszparitásra vonatkozóan.

Az alábbiakban az ilyen feltételeket és a rájuk illesztett modelleket ismertetem.
Nem minden modell került megvalósításra, néhol a vizsgálati eredményekből kiderült, nem reális vagy fölösleges részletes.

%-----------------------------------------------------------------------------------------------
\section{Modellek}\label{sect:Models}
%-----------------------------------------------------------------------------------------------

Ezen fejezet alapfeltevése, hogy a diszparitás kis régiókon belül nem változik tetszőleges.
Kis régió fogalom nagyon szubjektív, jelen állás szerint kísérletileg meghatározott fogalom.
A továbbiak a kép 16 pixel széles tartományát értem ezalatt.

Már korábban is feltettük\footnote{Az ablakméret megválasztásánál}, hogy nem számítunk tetszőlegesen nagy elmozdulásokra.
A modellezés során ennél erősebb feltételek keresése volt a cél.

Két esetet vizsgáltam a féléves munka során: lineáris modell és 2 konstans szakaszból álló modell.

%-----------------------------------------------------------------------------------------------
\subsection{Lineáris modell}\label{sect:linearModel}
%-----------------------------------------------------------------------------------------------

A lineáris modell esetében az volt a feltételezés, hogy diszparitás lineárisan változik rövid szakaszon.
Ekkor a modell \eqref{linModel} alakú.
\begin{equation}
y = a*x + b
\label{eq:linModel}
\end{equation}

Az illeszkedés hibája nyilvánvalóan igen nagy lesz, ha a diszparitásban ugrás\footnote{Ez a helyzet áll fenn például tárgyak pereménél} van a régión belül.
Kiegészíthető a modell úgy, hogy 2 egyenes szakaszt illesztünk az adatokra.
Így jóval több paraméter\footnote{Az egyenesek paraméterein túl a töréspont helyét is meg kell határozni.} meghatározása a feladat ugyan annyi bemenő adatpont használatával.
Ha ezt a problémát a régióméret növelésével próbáljuk csökkenteni, akkor pedig a modell alapfeltevése (lineáris változás) sérülhet.

Mélyebb vizsgálatnak nem vetettem alá ezt a modellt.
A jelenlegi pontossági igényeken belül még a lineáris változás is elhanyagolható kis szakaszokon.
Több mintát megvizsgálva arra jutottam, hogy a konstans modell is elegendő a feladat megoldására.

%-----------------------------------------------------------------------------------------------
\subsection{Konstans modell}\label{sect:constModel}
%-----------------------------------------------------------------------------------------------

Hasonlóan a lineáris modellhez, ebben az esetben is be meg kell engedni egy töréspontot modellben.
Így egy három paraméteres modell adódik:
\begin{itemize}[noitemsep]
\item Diszparitás a régió elején
\item Diszparitás a régió végén
\item Törés koordinátája
\end{itemize}

Még így is relatív kevés adat jut egy paraméterre (16 pixeles régióméret esetén).
A gyakorlati implementációban további optimalizálásként 2 esetet különböztetek meg.
Ha a két diszparitás paraméter eltérése igen kicsi (1-2 pixel), akkor 1 paraméteres modellt használok.

%-----------------------------------------------------------------------------------------------
\section{Illesztési módszerek}\label{sect:modelMatch}
%-----------------------------------------------------------------------------------------------

Az előző féléves munkám során a rekonstrukció eredményeként mindig a legjobb illeszkedési mutatóval rendelkező pont lett elfogadva.
A mostani módszerek ettől gyökeresen különböznek.
Az első maximumok gyakran hibás egyezés következményei, zajosak lehetnek.
Az vizualizált illeszkedési gráfokat vizsgálva szembetűnik, hogy sok esetben az első maximum ugyan hibás, de a helyes eredmény is detektálható lenne az adatokból.
Itt jön a képbe ezen fejezet tárgya: a lokális környezet.
Ha nem csak az első maximumot tároljuk, hanem az első néhányat, a szomszédos pontok diszparitása alapján kiválaszthatjuk a legvalószínűbb egyezést közülük.

Ezt a választást segítik az feljebb vázolt modellek.
Alább pedig két lehetőséget mutatok be paraméterek meghatározására.

%-----------------------------------------------------------------------------------------------
\subsection{Kumulált korreláció}\label{sect:CumCorr}
%-----------------------------------------------------------------------------------------------



%-----------------------------------------------------------------------------------------------
\subsection{RANSAC}\label{sect:ransac}
%-----------------------------------------------------------------------------------------------