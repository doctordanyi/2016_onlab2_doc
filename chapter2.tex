%-----------------------------------------------------------------------------------------------
\chapter{Teszt platform}\label{sect:Software}
%-----------------------------------------------------------------------------------------------

A munka kutatási jellege miatt fontos volt egy rugalmas, moduláris teszt platform létrehozása.
A programozáshoz a C\# nyelvet választottam a magas szintű funkciói miatt: igen fontos volt a gyors prototípus fejlesztés lehetősége.
A képfeldolgozási feladatokhoz az OpenCV 3 könyvtárat használtam, az EmguCV wrapperen keresztül.

Amint már többször hangsúlyozásra került, a modularitás és rugalmasság fontos szempont volt a keretrendszer tervezése során.
Ezt a rugalmasságot úgy értem el, hogy feldolgozási sorokba rendeztem az elvégzendő képfeldolgozási feladatokat.

Három sor került definiálásra az előző fejezetben ismertetett struktúrát szem előtt tartva.
Egy \emph{előfeldolgozási} sor, egy \emph{diszparitás számító} sor és egy \emph{utófeldolgozási} sor.
Köztük éles természetes határvonalként adódik a bemeneti argumentumaik mibenléte.
Az előfeldolgozó sor 1 képen értelmezett műveletek hajt végre a referencia- és az adatképen.
A diszparitás számító sor 2 képen operál és kimenetként egyet szolgáltat.
Az utófeldolgozó pedig ismét egy képen értelmezett műveleteket támogat, amiket a diszparitásképen hajt végre.



%-----------------------------------------------------------------------------------------------
\section{Programozói interfész}\label{sect:ProgInterface}
%-----------------------------------------------------------------------------------------------



%-----------------------------------------------------------------------------------------------
\section{Felhasználói felület}\label{sect:UserInterface}
%-----------------------------------------------------------------------------------------------


