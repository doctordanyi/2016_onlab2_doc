%-----------------------------------------------------------------------------------------------
\chapter{Teszt platform}\label{sect:Software}
%-----------------------------------------------------------------------------------------------

A munka kutatási jellege miatt fontos volt egy rugalmas, moduláris teszt platform létrehozása.
A programozáshoz a C\# nyelvet választottam a magas szintű funkciói miatt: igen fontos volt a gyors prototípus fejlesztés lehetősége.
A képfeldolgozási feladatokhoz az OpenCV 3 könyvtárat használtam, az EmguCV wrapperen keresztül.

Amint már többször hangsúlyozásra került, a modularitás és rugalmasság fontos szempont volt a keretrendszer tervezése során.
Ezt a rugalmasságot úgy értem el, hogy feldolgozási sorokba rendeztem az elvégzendő képfeldolgozási feladatokat.

Három sor került definiálásra az előző fejezetben ismertetett struktúrát szem előtt tartva.
Egy \emph{előfeldolgozási} sor, egy \emph{diszparitás számító} sor és egy \emph{utófeldolgozási} sor.
Köztük éles természetes határvonalként adódik a bemeneti argumentumaik mibenléte.
Az előfeldolgozó sor 1 képen értelmezett műveletek hajt végre a referencia- és az adatképen.
A diszparitás számító sor 2 képen operál és kimenetként egyet szolgáltat.
Az utófeldolgozó pedig ismét egy képen értelmezett műveleteket támogat, amiket a diszparitásképen hajt végre.

Ahol lehetett és szükséges volt, több szálon futni képes kódot írtam.
Ez ugyan extra erőfeszítés, de mivel gyakori volt az új algoritmusok tesztelése, ritka volt a hatékony implementáció.
A kézben tartható futásidőkhöz elengedhetetlen volt a párhuzamosítás.

Kihívást jelentett egy használható, intuitív grafikus felület tervezése.
Ez részben a rutin hiányának, részben pedig a sok, komplex adat átlátható megjelenítésének igényéből adódott.

%-----------------------------------------------------------------------------------------------
\section{Programozói interfész}\label{sect:ProgInterface}
%-----------------------------------------------------------------------------------------------

Az előző féléves munka során tapasztaltam, mekkora nehézséget tud okozni egy nem jól skálázódó, átgondolatlan program struktúra.
Az akkor elkövetett hibákból tanulva igyekeztem most egy egyszerűen bővíthető, jövőbeni ötletekre is felkészített keretet tervezni.

Első és legfontosabb teendőm volt a felhasználói felület és a feldolgozás szeparálása.
Ezzel is készülve arra, hogy esetleg a teljes back-end lecserélhető legyen, például egy C++-ban fejlesztett könyvtárra.

%-----------------------------------------------------------------------------------------------
\subsection{Adatstruktúra}\label{sect:dataStructure}
%-----------------------------------------------------------------------------------------------

A munka jellegéből adódóan szükségét éreztem, hogy a kimenet a feldolgozási lánc bármely szakaszában megjeleníthető legyen.
Ezért a minimálisan elvárhatótól több képet tárol a program a memóriában.
Ennek az emberi vizsgálatokon kívül a feldolgozás során is tapasztalható előnye.

A programban a következő szakaszait tárolja a feldolgozásnak:
\begin{itemize}[noitemsep]
	\item Nyers adat
	\item Nyers referencia
	\item Előfeldolgozott adat
	\item Előfeldolgozott referencia
	\item Nyers diszparitás
	\item Utófeldolgozott diszparitás
	\item Vizualizált diszparitás
\end{itemize}

Az utófeldolgozott és a vizualizált diszparitás szeparált tárolását az indokolja, hogy a vizualizáció során megváltozik az adat jellege: 3 csatornás RGB képet kapunk az egy csatornán tárolt értékek helyett.

Ezek az adatok nem függetlenek egymástól, bizonyos sorrendiségnek fel kell állnia köztük.
Adódik az igény, hogy a felhasználói felületen bizonyos funkciók csak akkor legyenek elérhetőek, ha a feldolgozás egy része már befejeződött.
Ennek támogatása céljából lekérhető a fenti pufferek állapota, feltétel állítható be rájuk.

%-----------------------------------------------------------------------------------------------
\subsection{Feldolgozási lépés}\label{sect:processingStep}
%-----------------------------------------------------------------------------------------------


%-----------------------------------------------------------------------------------------------
\section{Felhasználói felület}\label{sect:UserInterface}
%-----------------------------------------------------------------------------------------------


